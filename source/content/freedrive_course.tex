\section{Free Drive (w/o Obstacles)}

In this event, the vehicle shall automatically cover the farthest possible
distance in a given time. The vehicle drives in the right lane.

\subsection{Scenario}

The complexity of this scenario is limited. It consists of a road with two
parallel lanes - one for each driving direction. This scenario shall imitate a
rural road environment, consisting of long straight sections, tight turns,
intersections, side road junctions. The lanes are limited by different types of
lane markings. All markings are white and approx. 18 mm to 20 mm wide, if not
specified differently. The starting line (a checkered line of approx. 50 mm)
marks the beginning of the track.

\subsubsection{Lane width}

Each lane has a width of 350 mm to 450 mm, measured from the inside of the
respective markings. The left and right markings do not show lateral
misalignments. However, the centerline may under circumstances (e.g. because of
change of marking type, cf. next section) display lateral misalignments.

\subsubsection{Lane markings}
\label{lane_markings}

Both lanes are separated by a dashed center line. The center line is
interrupted every 200 mm for another 200 mm. This shape continues until
reaching an intersection or the starting line, so that the center line might
stop with a gap at these points.

Alternatively to the dashed center line, a double solid line can be present. In
this case the solid lines are spaced approx. 20 mm apart, yielding a total
marking width of approx. 56 mm to 60 mm. A combination of a solid and a dashed
line is also possible. In both cases, the inner edges of the markings define
the width of the lane. Marking types can occur in arbitrary order. Marking
types will persist for a distance of at least 1000 mm. There will be immediate
changes between marking types (cf. Section \ref{fig_road_layout}). For the Free
Drive event, these marking types are to be treated as regular dashed markings.

The left and right track boundaries are given by solid white lines. On straight
sections of the track, the outer track boundaries can also mark side road
junctions. In this case, the outer track boundaries are marked with 100 mm long
dashes, interrupted by 50 mm long gaps. These markings are to be treated as
solid lines and must not be crossed, as the vehicle is assumed to have the
right of way.

Side road junctions may be at most 960 mm long. The junction is only marked by
the change in marking types, there are no further markings for the side lane.
Neighboring sections of the track are space at least 50 mm apart, measured from
the outer edges of the markings. The minimal distance of the track to the end
of the course area is 300 mm. The sharpest turn has an inner radius of 1000 mm.

The circuit is mostly planar. Parts of the track can show slopes of up to 10\%
(0.1 m difference in height on a length of 1 m). Uphill and downhill grades
will be announced by traffic signs (cf. Section \ref{traffic_signs}). The signs
will be placed at least 1000 mm prior to any change of slope. All of the lane
markings can be missing at arbitrary locations for a maximum of 1000 mm. Except
for intersections, no more than two markings are missing at the same time.

An example scenario is depicted in Section \ref{fig_example_circuit} in the
appendix.

\HighlightNew{The vehicle has to stay in the right lane at all times. Crossing the lane markings with more than two wheels will result in a penalty.
	Exempt from this rule are road elements where the vehicle is expected to change lanes (i.e. intersections).}

In this event, no obstacles are located on the track. Possible stop lines and
regulations concerning the right of way are to be ignored.

\subsubsection{Traffic Signs}
\label{traffic_signs}

In addition to the the steep hill signs described above, other supporting
traffic signs can be present on the roadside.

Guide signs will be used to indicate sharp turns. They mark a curved section of
the track with radii below 1200 mm, if it is located after a straight section
of at least 3 m length. A first guide sign will be placed approx. 1.5 m before
the transition to the turn. The second sign marks the beginning of the turn.
Smaller signs will be repeated approximately every 400 mm until reaching the
apex of the turn.

Additional traffic signs can be present at the roadside. They are located on
the right-hand side of the lane. For an exact specification see Section
\ref{fig_traffic_signs}. In this event, regulations announced by traffic signs
can be ignored.

\subsubsection{Artifacts}
\label{artifacts}

The design of the area outside of the road is not defined. Artifacts in the
form of objects or remainders of lane markings might be located outside of the
road area. The minimum distance between artifacts and valid lane markings is
100 mm.

\subsection{Execution of the Event}

\subsubsection{Start}

The starting order of the teams will be announced by the commission, visualized
using the start scheduling system (cf. Section \ref{start_scheduling}) during
the competition. The vehicle must be placed in the start box, located next to
the track (cf. Section \ref{start_box}). The attempt is started by a judge or a
referee, signaled by the opening of the start box gate.

It is not strictly necessary to detect the presence of the markings on the
gate, the vehicle only has to detect when the gate is opened.(cf. Section
\ref{fig_start_box_markings})

\subsubsection{Attempts}

The attempt may be canceled while the gate of the start box is open. The team
is then allowed a second attempt, after all other teams have completed their
first attempt.\\ The commission may arbitrarily choose the time of the second
attempt if this is required to comply with the schedule. The affected team will
be given at least 10 minutes to prepare for the second attempt after being
informed by the commission. The vehicle does not have to be placed back at the
"parc fermé" and may be modified by the team while preparing for the second
attempt. Cancelling an attempt is penalized (cf. Section
\ref{freedrive_scoring}). A missed start results in a second attempt
automatically.

\subsection{RC-Mode}

In case the vehicle is not able to continue following the track on its own, the
team may activate RC-mode in order to get the vehicle back into normal
behavior. If the vehicle does not return into the right driving lane on its
own, RC-mode must be activated immediately. Distances travelled outside of the
driving lane will otherwise be subtracted from the total distance covered. Each
activation of RC-mode is penalized. RC-mode is subject to the regulations in
Section \ref{rc_mode}.

\subsection{Parking}

\HighlightNew{Parking has been moved to the Obstacle Evasion Course.}

\subsection{Scoring}
\label{freedrive_scoring}

The covered distance under consideration of penalties will be multiplied by the
achieved multiplier. The longest resulting distance will be awarded the maximum
number of points. The subsequent teams will be scored in relation to the best
team.

\subsubsection{Timing}

Each team has 2 min to complete this event. Timing for the event starts with
opening the start box gate described in Section \ref{start_box}.

\subsubsection{Penalties}
\label{freedrive_penalties}

\begin{table}[H]
	\begin{tabular}{@{}lcc@{}}
		\toprule
		\textbf{Violation}                                & \textbf{Maximum Count} & \textbf{Penalty} \\ \midrule
		Leaving the right lane                            & $\infty$               & 5m               \\
		Activation of RC-mode                             & $\infty$               & 5m               \\
		Faulty activation of the brake light              & 3                      & 2.5m             \\
		False usage of turn indicators                    & 2                      & 2.5m             \\
		Vehicle not placed inside the markings            & 2                      & 5m               \\
		Collision with obstacle                           & 6                      & 5m               \\
		Driving in the wrong direction at an intersection & $\infty$               & 5m               \\
		\bottomrule
	\end{tabular}
\end{table}

\subsubsection{Multipliers}
\label{freedrive_multipliers}

Each team starts this event with a multiplier of \textbf{1.0}.

\begin{table}[H]
	\begin{tabular}{@{}lcc@{}}
		\toprule
		\textbf{Triggering Event}         & \textbf{Maximum Count} & \textbf{Multiplier Modification} \\ \midrule
		Canceled attempt / second attempt & 1                      & -0.3                             \\
		WiFi enabled during competition   & 1                      & -0.5                             \\
		\bottomrule
	\end{tabular}
\end{table}