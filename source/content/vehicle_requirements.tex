\chapter{Vehicle Requirements and Limitations}

The observance of the following regulations will be monitored during the
competition. Violating these regulations will lead to a deduction of points or
exclusion from the competition. The same vehicle must be used for all events.

\section{Drivetrain}

The vehicle must be equipped with (an) electric motor(s). The number of driven
wheels is not limited (torque vectoring is allowed). Other motors (e.g.
combustion engines) are not permitted.

\section{Energy Supply}

Energy must be supplied in the form of batteries. Changing the batteries
between single events is allowed.

\section{Physical Dimensions}

The vehicles must be based on four-wheeled 1:10 scale chassis. Only two axles
are permitted. The wheelbase must measure at least 200 mm. The track width
(measured from the center of the wheels) must measure at least 160 mm. The
vehicle, including possible extensions and bodywork, must not be wider than 300
mm. The height of fixed installations must not exceed a height of 300 mm above
the track surface. Flexible antennae are allowed.

Apart from this, the design of the chassis is subject to the team’s creativity,
as long as it adheres the maximum physical dimensions. For the acceptance test,
the car must be able to drive through a fixed gate (inner dimensions: height
300 mm, width 300 mm) in RC-mode.

\section{Steering / Tires}

At least one axle must be steerable. Teams are expected to use cushion or foam
rubber tires. Other types of tires need to be confirmed by the commission prior
to the training sessions. The use of traction additives or studded tires is not
allowed.

\section{Sensor Setup}

The sensor setup can be arbitrarily chosen by the teams. Laser sensors are
allowed only up to class 2 devices.

\section{Data Transmission}

No data or signals must be transferred from the vehicle to the outside world
during the dynamic events, except for those signals necessary for the remote
control (cf. Section \ref{rc_mode}).\\ An active WiFi connection may be used
during dynamic events, this will lead to a decreased score multiplier.(cf.
Section \ref{freedrive_multipliers} and \ref{obstacle_scoring})

\section{WiFi-Network}

Each participating team may set up a single private wifi network.\\
For internet access, participants can connect to the eduroam network at the venue.\\

\section{Bodywork}

The teams must be able to quickly disassemble the vehicles’ bodywork, so that
the inner parts of the vehicle can be inspected at any time. The bodywork must
conform to IP 10 (EN 60529).

\section{RC-Mode}
\label{rc_mode}

In emergency situations, the vehicle must be stoppable and maneuverable using a
remote control. This can become necessary due to faults or errors in the data
processing or due to other problems so that the vehicle cannot continue to
execute its automated driving task.

\subsection{Activating RC-Mode}

RC-mode is activated by the remote control. An active RC-mode must be signaled
by utilizing a sufficiently bright, flashing, blue light, which is visible from
any position on the track. The light must be fixed at the highest point of the
vehicle. The light must flash with a frequency of 1 Hz, showing a duty cycle of
50\%, beginning with the status "on" when activating RC-mode. RC-mode must only
be activated after a clear misbehavior of the vehicle. This means e.g.
completely leaving the designated course of the track.

\subsection{Driving in RC-Mode}

Activation of RC-mode must instantly bring the vehicle to a complete halt,
without further steering maneuvers. The vehicle must be in standstill for at
least 1 s before it may be controlled with the remote control. During the
events, the vehicle must not drive faster than 0.3 m/s forward and backward
when RC-mode is engaged. However, the vehicle may be controlled directly after
having stopped during training. Additional functionality is not allowed in
RC-mode.

\subsection{Transmission Frequencies}

In order to limit interference between the vehicles of the different teams,
each team must inform the commission about the transmission frequency of their
remote control used when registering. Frequencies are issued on a
first-come-first-serve basis. Additionally, specific models are known to
interfere with Wi-Fi networks, or other infrastructure. Thus, remote controls
using frequencies in the 2.4 GHz band need to be confirmed by the commission
individually.

\section{Handling of the Vehicle}
\label{handling_vehicle}

\HighlightFix{Add navigation course or reformat}
The vehicle must provide two distinctive buttons (e.g. push-buttons,
touchscreen buttons, etc.), which start the different modes for the dynamic
events. The buttons must be \HighlightNew{located on the vehicle,} uniquely 
identifiable and easily reachable in order to allow non-team members 
(e.g. Judges, Referees) to start the vehicle.

\section{Lights}

As in real traffic, lights shall signal different driving maneuvers.

\subsection{Braking Lights}

Three clearly visible and differentiable braking lights must be installed at
the rear of the vehicle. Active braking must be signaled.

\subsection{Direction Indicators}

Each corner of the vehicle must be equipped with a yellow / orange light. The
respective lights at the correct side must be flashed at a maximum frequency of
2 Hz (50\% duty-cycle, initial state "on") when overtaking, turning, or
parking.

\subsection{RC-Mode-Indicator}

A clearly visible blue light is to be installed at the highest point of the
vehicle, which flashes to signal the activation of RC-mode (cf. Section
\ref{rc_mode}).

\section{Development Know-How}
\label{dev_know_how}

The basic concepts of the vehicle must be conceptualized and implemented by the
students themselves. They must not accept the direct help of professional
engineers or suppliers. The students are encouraged to do research and/or
discuss their problems with professional engineers or suppliers.

Ready-made solutions may never be included in the vehicle. This particularly
concerns the use of predesigned algorithms which may be part of a hardware
platform and serve the purpose of providing a fully functional system for
perception, behavior generation or control for automated vehicles or robots.

The final decision on acceptable components is taken by the commission. The
teams are encouraged to contact the commission early in case of doubts or
questions about a particular component. In case of violating these guidelines
or intentional fraud, the commission has the right to exclude the respective
team from the competition.

\section{Safety Regulations}

During the competition, safety instructions issued by the venue and commission
members are to be followed. Ignorance of notes or guidelines can be punished by
excluding the respective team from the training sessions or the competition.
Each individual is required at all times to take care that no other
participants are injured or other vehicles are damaged due to careless
behavior.

As far as the sensor setup is concerned, special requirements and restrictions
arise. All components within the vehicles must adhere to established guidelines
for safe public usage. Particularly the usage of active sensors can be limited
by this rule.

The teams must make sure that no third parties are subject to possible injury
due to installation or handling of the sensors. In case of questions concerning
particular sensors, the admission must be discussed with the commission prior
to the beginning of the training sessions.

Violations of these regulations lead to an immediate exclusion from the
competition. Any claim for compensation from the commission is excluded.

\section{Modification of the Vehicle}

During the dynamic events, the hardware of the vehicle must not be modified
except in case of supervised repair. The software must not be modified during
the dynamic events. Changing and charging batteries is allowed.