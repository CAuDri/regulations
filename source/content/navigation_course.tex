
\begin{highlight}[yellow]

    \section{Navigation Course}
    \label{navigation_course}

    The Navigation Course is a new event in 2025. The event is designed to test the
    ability of the vehicle to map an unknown environment and navigate between
    predefined landmarks.

    \subsection{Scenario}

    The track layout of the Navigation Course will not be announced before the
    event. All road elements of the Obstacle Evasion Course (cf. Section
    \ref{elements_obstacle_evasion}) can appear. No pedestrians and no static and
    dynamic obstacles will be present on the track. In this event, regulations
    announced by regular traffic signs and mandatory turning directions at
    intersections can be ignored.

    \subsubsection{Speed Limits}

    The track might include suburban areas (cf. Section \ref{suburban_area}) where
    a speed limit is enforced. The speed limit within the suburban section, as
    indicated by the traffic signs and road markings has to be scaled by 1:10. In
    addition to the speed limits depicted in the signs, marking the suburban
    scenario, other numeric signs in steps of 10 km/h might appear (e.g. a speed
    limit of 20 km/h). Speed limit zones begin and end at the road markings, as
    depicted in Section \ref{fig_speed_limit_zone}.

    \subsubsection{Landmarks}
    \label{landmarks}

    Landmarks are additional \"traffic signs" serving as navigation points for the
    vehicle. Each landmark will depict a QR code with a unique identifier.

    \HighlightChange{The landmarks have the same shape and size as a zone 30 
    traffic sign with the QR-Code in the middle} (cf. Section \ref{fig_landmarks}).
    Landmarks can appear anywhere on the track, where their position is not 
    obstructed by other road elements or obstacles. Each landmark will have two 
    traffic signs with the same identifier, one on each side of the track, facing 
    the vehicle on the right-hand side. This allows the vehicle to detect the 
    landmark from either direction.

    The QR codes will encode a unique identifier for each landmark. The identifiers
    will be in ascending order, starting with 1. During the event the vehicle must
    navigate to the landmarks in the given order.

    A maximum of 10 landmarks will be placed on the track. The exact number of
    landmarks will not be announced before the event.

    \subsection{Execution of the Event}

    The event can be divided into two parts, a mapping phase and a navigation
    phase.

    \subsubsection{Start}

    The starting order of the teams will be announced by the commission, visualized
    using the start scheduling system (cf. Section \ref{start_scheduling}) during
    the competition. The vehicle must be placed in the start box, located next to
    the track (cf. Section \ref{start_box}). The attempt is started by a judge or a
    referee, signaled by the opening of the start box gate.\\ It is not strictly
    necessary to detect the presence of the markings on the gate, the vehicle only
    has to detect when the gate is opened.(cf. Section
    \ref{fig_start_box_markings})

    \subsubsection{Attempts}

    The attempt may be canceled during the mapping phase while the gate of the
    start box is open. The team is allowed a second attempt, after all other teams
    have completed their first attempt. \newline The commission may arbitrarily
    choose the time of the second attempt if this is required to comply with the
    schedule. The affected team will be given at least 10 minutes to prepare for
    the second attempt after being informed by the commission. The vehicle does not
    have to be placed back at the "parc fermé" and may be modified by the team
    while preparing for the second attempt. Cancelling an attempt is penalized (cf.
    Section \ref{navigation_scoring}). A missed start results in a second attempt
    automatically. Any information about the track and the landmarks gathered
    during the first attempt may not be used in the second attempt and needs to be
    deleted from the vehicle.

    \subsubsection{Mapping Phase}

    The mapping phase will start when the start box gate is opened for the first
    time. The vehicle will be given 3 minutes to freely drive around the track and
    map the environment. Landmarks will be placed at random locations on the track.
    Information gathered during the mapping phase can later be used in the
    navigation phase.

    The end of the mapping phase will be signaled by the start scheduling system
    (cf. Section \ref{start_scheduling}) and the vehicle must be placed back at the
    start box. A button on the vehicle may be used to change the mode from mapping
    to navigation, it follows the same regulations as the buttons for the dynamic
    events (cf. Section \ref{handling_vehicle}). No further modifications or
    adjustments to the vehicle are allowed.

    \subsubsection{Navigation Phase}

    The navigation phase starts when the start box gate is opened for the second
    time during the event. The goal is to navigate between the landmarks in the
    correct order, covering the shortest distance possible.

    The order is given by the unique identifier in the QR codes of the landmarks
    (cf. Section \ref{landmarks}). Landmarks do not have to be visited strictly in
    this order (e.g. 1-2-3), but can be passed multiple times or in between (e.g.
    3-1-2-2-1-3) However, a pass is only counted if the previously required
    landmarks have already been visited.

    A landmark is considered visited the moment the vehicle has fully passed both
    traffic signs. The vehicle is not required to stop at the landmark or signal
    its arrival. The vehicle may pass the landmark from either direction.

    The navigation phase ends after the vehicle has visited all landmarks in the
    correct order. Each team will be given a maximum of 3 minutes to complete the
    navigation phase.

    \subsection{RC-Mode}

    In case the vehicle is not able to continue following the track on its own, the
    team may activate RC-mode in order to get the vehicle back into normal
    behavior. If the vehicle does not return into the right driving lane on its
    own, RC-mode must be activated immediately. Distances travelled outside of the
    driving lane will otherwise be added to the total distance covered.

    Bonuses cannot be earned by vehicle behavior shown in RC-mode. Passing a
    landmark in RC-mode will not be counted as a successful visit. RC-mode may not
    be used to influence the vehicle's navigation behavior in any way. Using
    RC-mode to correct the vehicle's path will result in a penalty (cf. Section
    \ref{navigation_scoring}). Each activation of RC-mode will be penalized.
    RC-mode is subject to the regulations in Section \ref{rc_mode}.

    \subsection{Scoring}
    \label{navigation_scoring}

    There will be no scoring for the Navigation Course in 2025.

    The total distance covered by the vehicle during the navigation phase will be
    measured. A shorter distance is considered better.

\end{highlight}
