{
\renewcommand*\footnoterule{} % No line above footnotes
\newcommand{\topstrut}{\rule{0pt}{3.5ex}}
\rowcolors{0}{white}{lightgray!40} % Alternate row colors

\subsection{Scoring}
Each team will start the event with a fixed number of base points. The base
points are predetermined by the commission and will depend on the length of the
track and the number of occuring elements. The total distance covered by the
vehicle will have no influence on the scoring.

\subsubsection{Timing}
Each team will be given a \textbf{5-minute time limit} to complete
\textbf{three laps} around the track. After the vehicle has completed all three
laps, the attempt is over and no further points will be awarded. Timing of the
event starts with the opening of the gate of the start box, described in
section \ref{start_box}.

\subsubsection{Evaluation}
When the vehicle passes one of the elements described in section
\ref{elements_obstacle_evasion} and \ref{elements_suburban_scenario}, it will
receive either a positive, neutral or negative evaluation.

Any time the vehicle complies with all the requirements of a certain element,
it will receive a positive evaluation and gain points.\\ If the vehicle fails
to comply with any of the requirements, it will generally receive a neutral
evaluation and no points will be awarded.\\ In some cases, the vehicle may
receive a negative evaluation and points will be deducted.

The following section \ref{obstacle_scoring_guidelines} provides an overview of
the scoring guidelines for each element. The bottom row of each table shows the
amount of points that will be awarded or deducted for each evaluation.

The final score of each team will be the sum of the base points and the points
gained or lost during the event. Only after a vehicle has completed at least
one full lap will the attempt be valid and any points be awarded. In case the
vehicle has left the track or skipped certain parts of it, the commission will
decide whether the attempt was valid.

\section{Scoring Guidelines}
\label{obstacle_scoring_guidelines}

\subsection*{Static Obstacles}
\begin{table}[H]
    \begin{tabularx}{\textwidth}{XXX}
        \toprule
        \textbf{Positive}                & \textbf{Neutral}             & \textbf{Negative}          \\
        \midrule
        Use of turn indicators           & Wrong use of turn indicators & Collision with an Obstacle \\
        Successfully passed the Obstacle & Merging distance > 2m        &                            \\
                                         &                              &                            \\
        \topstrut
        \textbf{+10}                     & \textbf{0}                   & \textbf{-10}               \\
        \bottomrule
    \end{tabularx}
\end{table}

\subsubsection*{Dynamic Obstacles}
\begin{table}[H]
    \begin{tabularx}{\textwidth}{XXX}
        \toprule
        \textbf{Positive}                & \textbf{Neutral}                & \textbf{Negative}          \\
        \midrule
        Use of turn indicators           & Wrong use of turn indicators    & Collision with an Obstacle \\
        Successfully passed the Obstacle & Merging distance > 2m           &                            \\
                                         & Obstacle passed in intersection &                            \\
        \topstrut
        \textbf{+15}                     & \textbf{0}                      & \textbf{-10}               \\
        \bottomrule
    \end{tabularx}
\end{table}

\subsubsection*{Intersections of the Rural Road Scenario}
\begin{table}[H]
    \begin{tabularx}{\textwidth}{XXX}
        \toprule
        \textbf{Positive}                                           & \textbf{Neutral}                                                   & \textbf{Negative}                                              \\
        \midrule
        Vehicle went straight through the intersection              & Vehicle made a wrong turn                                          & Collision with an Obstacle                                     \\
        \textit{Vehicle stopped at the stop line}\footnotemark[1]   & \textit{Vehicle stopped for < 3s}\footnotemark[1]                  & \textit{Vehicle did not stop at the stop line}\footnotemark[1] \\
                                                                    & \textit{Distance from stop-line > 15cm}\footnotemark[1]            &                                                                \\
        \textit{Vehicle respected the right-of-way}\footnotemark[2] & \textit{Vehicle did not respect the right-of-way}\footnotemark[2]  &                                                                \\
                                                                    & \textit{Obstacle has not cleared the intersection}\footnotemark[2] &                                                                \\
        \topstrut
        \textbf{+10}                                                & \textbf{0}                                                         & \textbf{-10}                                                   \\
        \bottomrule
    \end{tabularx}
\end{table}

\subsubsection*{Extended Intersections}
\begin{table}[H]
    \begin{tabularx}{\textwidth}{XXX}
        \toprule
        \textbf{Positive}                                           & \textbf{Neutral}                                                   & \textbf{Negative}                                      \\
        \midrule
        Vehicle stopped at the stop or give-way line                & \textit{Distance from stop-line > 15cm}\footnotemark[1]            & \textit{Did not stop at the stop line}\footnotemark[1] \\
                                                                    & \textit{Stopped for < 3s at stop line}\footnotemark[1]             &                                                        \\
                                                                    & \textit{Stopped for < 1s at give-way line}\footnotemark[3]         &                                                        \\
        \textit{Vehicle respected the right-of-way}\footnotemark[2] & \textit{Vehicle did not respect the right-of-way}\footnotemark[2]  & Collision with an Obstacle                             \\
                                                                    & \textit{Obstacle has not cleared the intersection}\footnotemark[2] &                                                        \\
        \textit{Vehicle took the right turn}\footnotemark[4]        & \textit{Vehicle took the wrong turn}\footnotemark[4]               &                                                        \\
                                                                    &                                                                    &                                                        \\
        \topstrut
        \textbf{+20}                                                & \textbf{0}                                                         & \textbf{-10}                                           \\
        \bottomrule
    \end{tabularx}
\end{table}

\footnotetext[1]{In case a stop line is present}
\footnotetext[2]{In case a dynamic Obstac is present}
\footnotetext[3]{In case a give-way line is present}
\footnotetext[4]{In case the intersection has a mandatory turn}

\subsubsection*{No-Passing Zones}
\begin{table}[H]
    \begin{tabularx}{\textwidth}{XXX}
        \toprule
        \textbf{Positive}                & \textbf{Neutral}                   & \textbf{Negative}          \\
        \midrule
        Vehicle stayed in the right lane & Vehicle started a passing maneuver & Collision with an Obstacle \\
        Distance to Obstacle > 30cm      & Distance to Obstacle < 30cm        &                            \\
                                         &                                    &                            \\
        \topstrut
        \textbf{+5}                      & \textbf{0}                         & \textbf{-10}               \\
        \bottomrule
    \end{tabularx}
\end{table}

\subsubsection*{Barred Area}
\begin{table}[H]
    \begin{tabularx}{\textwidth}{XXX}
        \toprule
        \textbf{Positive}                                           & \textbf{Neutral}                                                  & \textbf{Negative}          \\
        \midrule
        Vehicle did not enter the barred area                       & Vehicle entered the barred area                                   & Collision with an Obstacle \\
        \textit{Vehicle respected the right-of-way}\footnotemark[1] & \textit{Vehicle did not respect the right-of-way}\footnotemark[1] &                            \\
        Use of turn indicators                                      & Wrong use of turn indicators                                      &                            \\
        \topstrut
        \textbf{+15}                                                & \textbf{0}                                                        & \textbf{-10}               \\
        \bottomrule
    \end{tabularx}
\end{table}

\subsubsection*{Crosswalk}
\begin{table}[H]
    \begin{tabularx}{\textwidth}{XXX}
        \toprule
        \textbf{Positive}                                               & \textbf{Neutral}                                                   & \textbf{Negative}           \\
        \midrule
        \textit{Vehicle stopped at the crosswalk}\footnotemark[2]       & \textit{Vehicle stopped for < 3s at the crosswalk}\footnotemark[2] & Collision with a pedestrian \\
        \textit{Pedestrians have cleared the crosswalk}\footnotemark[2] & \textit{Stopping distance from crosswalk > 15cm}\footnotemark[2]   &                             \\
                                                                        & \textit{Pedestrians have not fully crossed}\footnotemark[2]        &                             \\
        \topstrut
        \textbf{+15}                                                    & \textbf{0}                                                         & \textbf{-10}                \\
        \bottomrule
    \end{tabularx}
\end{table}

\footnotetext[1]{In case a dynamic Obstacle is present}
\footnotetext[2]{In case a pedestrian is present}

\subsection*{Additional Penalties}
\begin{table}[H]
    \begin{tabular}{@{}lccc@{}}
        \toprule
        \textbf{Violation}            & \textbf{Maximum Count} &  & \textbf{Penalty}   \\
        \midrule
        Second Attempt                & 1                      &  & -0.5 x base points \\
        Active WiFi Connection        & 1                      &  & -0.5 x base points \\
        Exceeding Speed-Limit         & $\infty$               &  & -10                \\
        Activation of RC-Mode         & $\infty$               &  & -5                 \\
        Leaving the right lane        & \colorbox{red}{10}     &  & -5                 \\
        Collision with road sign      & $\infty$               &  & -5                 \\
        Falsely using turn indicators & \colorbox{red}{10}     &  & -5                 \\
        \bottomrule
    \end{tabular}
\end{table}
\clearpage
}
