\chapter{Overview}

\section{Objectives}

The student competition “Cognitive Autonomous Driving (CAuDri) Challenge”
provides a platform for student teams to get involved with the
conceptualization and implementation of automated model vehicles. The challenge
is to realize the best performing vehicle guidance system for different
scenarios, which have been derived from requirements arising from a realistic
environment.

In the annual competition, participating students have the opportunity to
present their know-how while competing with teams from other universities.

\section{Tasks}

The student team is put in charge of developing, producing and demonstrating a
cost- and energy-efficient 1:10 concept for an automated vehicle by a fictional
OEM. During the competition several driving tasks have to be executed as fast
and precise as possible. In addition, the developed concept must be presented
and explained.

\begin{highlight}
	\section{Scoring}

	Each team will perform in multiple dynamic events, challenging different
	aspects of the vehicle's capabilities. Each event is scored individually.
	\newline
\end{highlight}

% The maximum amount of points is distributed to the different events as follows:

% \subsection{Dynamic Events}

% \begin{table}[h]
% 	\begin{tabular}{|l|l|}
% 		\hline
% 		Free Drive:  & 300 points \\ \hline
% 		Obstacle Evasion Course: & 400 points \\ \hline\hline
% 		Maximum Total Score:     & 700 points \\ \hline
% 	\end{tabular}
% \end{table}

